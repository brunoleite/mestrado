\section{Related Work}


% 1) Efforts on network evolution. 

There has been a number of recent efforts that attempt to analyze and model and community structure and network evolution.  Viswanath \textit{et al.}~\cite{Viswanath:2009} study
the evolution of activity between users in Facebook and found that that links in the activity network tend to come and go rapidly over time, and the strength of ties exhibits a
general decreasing trend of activity as the social network link ages.

Kumar \textit{et al.}~\cite{Kumar:2006} 
\cite{Backstrom:2006} \\
\cite{Patil:2012} \\
\cite{Leskovec:2005} \\
\cite{Leskovec:2008} \\
\cite{Wu:2009} \\

Scientific communities. 
\cite{Lopes:2011} \\
\cite{Huang:2008} \\

Games
\cite{Ducheneaut:2007} \\

% 2) efforts that compute the k-core and try to find community leaders.  
% Diferença. Aqui nós olhamos as caracteristicas o impacto do core nas caracteristicas estruturais. 

When it comes to identifying the community core, there are many approaches that extract the core based on structural properties of the graph. 
Particular, 


\cite{Sachan:2012} \\
\cite{Seifi:2012:CCE:2187980.2188258}
Here we take a different approach by measuring the 

% 2) Varias artigos que visam 
Is showen in another study \cite{Kumar:2006} the components evolutions of two social networks of a big company. These 
study analysis the components of three ways. The analysis is done about nodes that have degree one, intermidiate components 
(components that are not the gigant component and has more than one degree) and the gigant component. In the intermidiate 
components is showed the concept of star nodes, this nodes are importants to the social networks, because they can make 
the social network grow. Also proposed is mathematic model that making the prediciton of behavior identified in these 
study. In this study \cite{Kumar:2006} is done a characterize about the nodes type too where they can be passive, 
linkers and invites.
\\



