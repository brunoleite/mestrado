Diversos esforços têm sido realizados para compreensão e modelagem de
aspectos dinâmicos de comunidades científicas. Apesar do grande interesse, pouco se sabe
sobre o papel que diferentes membros têm na formação da estrutura topológica da rede 
dessas comunidades. Nesta dissertação, investigamos o papel 
que os membros do núcleo de comunidades científicas têm na formação e evolução de sua rede de 
colaboração. Para isso, definimos o núcleo de uma comunidade com base em uma 
métrica, denominada \textit{CoScore}, derivada do índice~h que captura 
tanto a prolificidade quanto o envolvimento dos pesquisadores na comunidade. 
Nossos resultados subsidiam uma série de observações importantes relacionadas à 
formação e aos padrões de evolução das comunidades. Particularmente, mostramos que os membros 
do núcleo das comunidades atuam como pontes que conectam pequenos grupos de pesquisa. 
Além disso, esses membros são responsáveis pelo aumento do grau médio de toda a rede
que representa a comunidade e pela redução de sua assortatividade. Mais importante, 
notamos que variações no conjunto de membros que compõem o núcleo das comunidades tendem a ser fortemente 
correlacionadas com variações dessas métricas. Mostramos ainda que nossas observações 
são importantes para caracterizar o papel dos principais membros na formação e na estrutura das comunidades.

% Resumo para o poster
% Diversos esforços têm sido realizados para compreensão e modelagem de aspectos dinâmicos 
% de comunidades científicas. Apesar do grande interesse, pouco se sabe sobre o papel que 
% diferentes membros têm na formação da estrutura topológica da rede dessas comunidades. 
% Nesta dissertação, investigamos o papel que os membros do núcleo de 
% comunidades científicas têm na formação e evolução da estrutura de sua rede de colaboração. 
% Para isso, definimos uma métrica, denominada CoScore, que permite a definição do núcleo 
% dessas comunidades. Nossos resultados subsidiam uma série de observações importantes 
% relacionadas à formação e ao padrões de evolução das comunidades. Particularmente, 
% mostramos que variações no conjunto de membros do núcleo das comunidades tendem a ser 
% fortemente correlacionadas com variações na topologia da rede. Mostramos que nossas 
% observações são importantes para caracterizar o papel dos principais membros na formação 
% e na estrutura das comunidades.

% \keywords{Comunidades Científicas, Núcleos de Comunidades, Evolução de Comunidades}

% There have been considerable efforts in the literature towards understanding and modeling dynamic aspects of 
% scientific communities. Despite the great interest, little is known about the role that different members play 
% in the formation of the underlying network structure of such communities. In this paper, we provide a wide 
% investigation of the roles that members of the core of scientific communities play in the collaboration network 
% structure formation and evolution. To do that, we define a community core based on an individual metric, 
% \textit{core score}, which is an h-index derived metric that captures both, the prolificness and the involvement
% of researchers in a community. Our results provide a number of key observations related to community formation and 
% evolving patterns. Particularly, we show that members of the community core work as bridges that connect smaller 
% clustered research groups. Furthermore, these members are responsible for an increase in 
% the average degree of the whole community underlying network and a decrease on the overall network assortativeness. 
% More important, we note that variations on the members of the community core tend to be strongly correlated with 
% variations on these metrics. We argue that our observations are important for shedding a light on the role of key 
% members on community formation and structure.

