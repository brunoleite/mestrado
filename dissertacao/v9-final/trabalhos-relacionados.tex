%%%%%%%%%%%%%%%%%%%%%%%%%%%%%%%%%%%
\section{Trabalhos Relacionados}
%%%%%%%%%%%%%%%%%%%%%%%%%%%%%%%%%%%
% \redcomment{Revisar tradução e texto. Adicionar mais trabalhos e detalhar melhor alguns.}

Recentemente, esforços tem sido feitos com o intuito de analisar a estrutura das comunidades e a evolução de suas redes.
Particularmente, \cite{Kumar2006} analisaram duas grandes redes através do tempo, Flickr\footnote{http://www.flickr.com/} e Yahoo! 360, para encontrar uma 
segmentação dessas redes em indivíduos isolados, componentes intermediários, que também são comunidades isoladas, e o maior 
componente conectado. Nesse trabalho, os autores identificaram que alguns componentes possuem um centro constituído 
de nodos de maior grau com muitos nodos de menor grau conectados a eles e que foram caracterizados 
estrelas, isto é, componentes que são formados rapidamente, mas não são absorvidos pelo maior componente conectado.
Assim, eles propuseram um modelo de crescimento capaz de gerar redes com características similares às das redes estudadas.
Este modelo se baseia em três tipos de nodo: passivos, recrutadores e agregadores. Usuários passivos se juntam à rede pela
curiosidade ou pela insistência de amigos, mas nunca se comprometem efetivamente em atividades da rede. Recrutadores estão
interessados em transformar comunidades \textit{offline} em comunidades \textit{online} e recrutar seus amigos a participarem 
da rede. Por fim, os agregadores são participantes que contribuem ativamente para o crescimento da rede e frequentemente se conectam a outros 
membros semelhantes.

\cite{Ducheneaut2007} extraíram e caracterizaram explicitamente, comunidades criadas a partir de cinco servidores do 
\textit{World of Warcraft}, um jogo multijogador massivo. O estudo é centralizado em grupos de jogadores 
formados para realizar atividades em conjunto, denominados guildas. Tais grupos apresentam dificuldades 
consideráveis de administração e muitos não sobrevivem ao longo do tempo, sendo os líderes membros importantes para a 
longevidades das guildas. Os autores apresentaram algumas propriedades demográficas sobre esses grupos, 
tais como, distribuição do tamanho das guildas, interação de membros e não membros de guildas em eventos 
do jogo, e tamanho das guildas ao longo do tempo. Em seguida, mostraram o impacto das guildas na estrutura da 
rede utilizando métricas clássicas de redes complexas como centralidade e densidade. Eles sugeriram ainda
um painel para monitorar o tamanho, número de subgrupos e densidade das guildas, além de uma visualização da topologia 
da rede.

Complementarmente ao estudo de \cite{Ducheneaut2007}, \cite{Patil2012} analisaram e modelaram fatos que levam usuários a deixar 
ou entrar em guildas do jogo \textit{World of Warcraft}. Nesse trabalho os autores propõem um modelo capaz de predizer se e 
quando um jogador irá deixar a comunidade e também qual o impacto dessa saída. Este modelo é baseado em 
várias características, tais como, níveis dos jogadores e das guildas, atividades do jogo e características sociais. Estas 
características sociais estão ligadas a um jogador podendo ser o seu número de amigos, o número de amigos que já deixaram a comunidade, 
e a percentagem de membros na comunidade que interagem com o jogador em questão, bem como a frequência. O modelo construído se mostrou
efetivo em predizer quando os usuários deixarão suas comunidades e quando essas comunidades irão se desfazer na rede.

\cite{Viswanath2009} estudaram dois tipos de rede formados a partir de dados do Facebook\footnote{http://www.facebook.com}. A primeira rede foi construída utilizando 
os laços de amizade entre os usuários e a segunda utilizando as interações entre os usuários, e.g., postagens em mural e seus 
comentários. Eles constataram que muitos usuários aceitam pedidos de amizade por cortesia, de modo que a rede baseada
em laços de amizades apresenta um crescimento surreal. A rede construída a partir das interações entre os usuários possui comportamento 
diferente. Nessa rede, os laços tendem a surgir e desaparecer rapidamente através do tempo, e a força dos laços apresenta, de forma geral, 
uma tendência decrescente das atividades que acompanham a idade dos \textit{links} nas redes sociais. O trabalho também apresenta a métrica 
\textit{resemblance}, utilizada para medir a sobreposição entre duas instâncias da rede, para mostrar que a rede possui um núcleo que
persiste através do tempo. Finalmente, os autores apresentam uma análise utilizando métricas clássicas de redes complexas, mostrando
que as propriedades estruturais das redes estudadas não apresentam grandes variações ao longo do tempo.

% Recentemente, esforços tem sido feitos com o intuito de analisar a estrutura das comunidades e a evolução de suas redes. 
% Particularmente, \cite{Kumar2006} analisaram duas grandes redes para encontrar uma segmentação dessas redes
% em indivíduos isolados, comunidades isoladas e o maior componente conectado. Assim, eles propuseram um modelo de crescimento das
% redes apto a gerar redes com características similares a das redes estudadas. \cite{Ducheneaut2007} extraíram e caracterizaram 
% explicitamente, comunidades criadas a partir de \textit{World of Warcraft}, um jogo multijogador massivo.
% Complementarmente, \cite{Patil2012} analisaram e modelaram fatos que levam usuário a deixarem ou entrarem
% em comunidades de jogos online. \cite{Viswanath2009} estudaram a evolução das atividades entre 
% usuário no Facebook e encontraram que alguns links em atividades na rede tendem a surgir e desaparecer rapidamente através do tempo, e
% a força dos laços apresentam, de forma geral, uma tendência decrescente das atividades que acompanham as idades dos links nas 
% redes sociais.
% There has been a number of recent efforts that attempt to analyze community structure and network evolution.  Particularly, Kumar \textit{et al.}~\cite{Kumar:2006} analyzed two large networks to
% find a segmentation of these networks into singletons, isolated communities, a giant component. Then, they propose a network growth model able to generate networks with similar
% characteristics.  Ducheneaut \textit{et al.}~\cite{Ducheneaut:2007} extracted and characterized explicitly created communities from the World of Warcraft, a massive multiplayer game.
% Complementarily, Patil \textit{et al.}~\cite{Patil:2012} analyzed and modeled factors that make users to leave or join on-line gaming communities.  Viswanath \textit{et
% al.}~\cite{Viswanath:2009} studied the evolution of activity between users in Facebook and found that that links in the activity network tend to come and go rapidly over time, and
% the strength of ties exhibits a general decreasing trend of activity as the social network link ages.

Em termos de modelos para redes sociais dinâmicas, \cite{Leskovec2005} investigaram nove redes, que incluem redes de citações de 
diferentes áreas da física, citações de patentes americanas e afiliações de pesquisadores, para mostrar 
que essas redes densificam ao longo do tempo, com o número de arestas crescendo de forma superlinear em relação
ao número de nodos e que o caminho mínimo médio entre os nodos geralmente diminui através do tempo. Baseado nessas observações, eles 
desenvolveram um modelo de geração de redes que incorpora tais propriedades, chamado de \textit{Forest Fire}. Esse modelo necessita
apenas de dois parâmetros e é capaz de capturar padrões como a lei de potência da densificação e redução do diâmetro. Mais recentemente, 
\cite{Leskovec2008} apresentaram um estudo detalhado da evolução de redes por meio da análise de quatro grandes 
redes sociais \textit{online}, são elas: Flickr, Delicious\footnote{http://delicious.com/}, 
Answers\footnote{http://www.answers.com/} e LinkedIn\footnote{http://www.linkedin.com}. Eles investigaram uma grande variedade de estratégias de formação 
de redes para mostrar que a localização das arestas desempenha um papel crítico na evolução dessas redes. Baseados nessas observações, 
os autores desenvolveram um modelo de evolução em que novos nodos surgem na rede com uma taxa predeterminada. O modelo também 
considera que novas arestas possuem distâncias muito curtas, normalmente formando triângulos, sendo assim, o modelo segue este padrão para 
formação de novas arestas. Diferentemente dos esforços acima referidos, nosso trabalho foca em propriedades das comunidades e no papel que 
os líderes dessas comunidades desempenham na topologia da rede.

% Em termos de modelos para redes sociais dinâmicas, \cite{Leskovec2005} investigaram uma grande número 
% de grafos para mostrar que grafos densificam ao longo do tempo, com o número de arestas crescendo de forma superlinear em relação
% ao número de nodos e a distância média entre os nodos geralmente diminuem através do tempo. Baseado nessas observações, eles 
% desenvolveram um modelo de geração de grafos que incorpora tais propriedades. Mais recentemente, \cite{Leskovec2008} 
% apresentaram um estudo detalhado de evoluções de redes pela análise de quatro grandes redes sociais online. Eles investigaram uma grande
% variedade de estratégias de formações de redes para mostrar que a localização das arestas desempenham um papel crítico na evolução das redes.
% Baseado nessas observações, eles desenvolveram um modelo de evolução de redes, em que nodos alcançam uma taxa predeterminada. Diferentemente
% dos esforços acima referidos, nosso trabalhos foca em propriedades das comunidades e no papel que os líderes das comunidades desempenham
% na topologia da rede.
% In terms of models for network dynamics, Leskovec \textit{et al.}~\cite{Leskovec:2005} investigated a wide range of real graphs to show that graphs densify over time, with the
% number of edges growing super linearly in the number of nodes and that the average distance between nodes often shrinks over time. Based on these observations, they develop a graph
% generation model that incorporates such properties.  More recently, Leskovec \textit{et al.}~\cite{Leskovec:2008} presented a detailed study of network evolution by analyzing four
% large on-line social networks.  They investigated a wide variety of network formation strategies to show that edge locality plays a critical role in the evolution of networks. Based on
% this observation, they developed a model of network evolution, in which nodes arrive at a pre-specified rate.  Differently from the above efforts, our work focuses on community
% properties and the roles that community leaders play in the underlying network structure.

Alguns outros trabalhos também abordaram o estudo de comunidades científicas. Particularmente, \cite{Backstrom2006} 
estudaram comunidades na LiveJournal\footnote{http://www.livejournal.com/} e comunidades científicas extraídas 
da DBLP. Esses dois conjuntos de dados possuem comunidades explícitas, onde 
conferências representam comunidades na DBLP. Os autores identificaram que as tendências que levam 
indivíduos a entrar em comunidades e comunidades a crescer rapidamente dependem sutilmente da topologia 
da rede. Por exemplo, a tendência de um indivíduo a entrar em uma comunidade é influenciada não só pelo número 
de amigos que ele possui dentro daquela comunidade, mas também de como esses amigos estão conectados entre si. Eles utilizaram 
técnicas de árvore de decisão para identificar as estruturas mais significativas que afetam estas comunidades, 
e desenvolveram uma nova metodologia capaz de mensurar a mudança de indivíduos entre comunidades 
e também verificar como isto está relacionado com mudanças de interesse dentro das comunidades.

\cite{Huang2008} usaram os dados da biblioteca digital CiteSeer\footnote{http://citeseer.ist.psu.edu} 
para construir uma rede de coautoria em Ciência da Computação abrangendo pesquisas realizadas entre 1980 e 
2005. Eles realizaram estudos sobre tendências de evolução das colaborações sob duas perspectivas, rede completa 
e comunidades. Entre as suas principais observações, eles mostraram que a área de Ciência da Computação apresenta 
padrões de colaboração mais similares à Matemática do que à Biologia. Além disso, eles também quantificaram 
e compararam padrões de colaboração de seis comunidades dentro da área da Ciência da Computação: 
Inteligência Artificial, Aplicações, Arquitetura, Bancos de Dados, Sistemas e Teoria. Com base nessas 
informações, os autores propuseram um modelo de aprendizagem e predição de colaborações entre pares de autores.
Diferente desses esforços, nesta dissertação focamos no estudo das propriedades do núcleo da comunidade, de modo que 
nossas análises são complementares a essas.

% Existem também esforços que tentaram estudar comunidades científicas. Particularmente, \cite{Backstrom2006} 
% estudaram comunidades na LiveJournal~\footnote{http://www.livejournal.com/} e comunidades científicas extraídas da DBLP~\footnote{http://dblp.uni-trier.de/} 
% para encontrar que a tendência dos indivíduos entrarem nas comunidades e das comunidades de crescerem rapidamente, depende sutilmente da topologia da rede. 
% \cite{Huang2008} usaram os dados da DBLP para construir a rede do campo de Ciência da Computação cobrindo a colaboração de pesquisa de 1980 a 2005. Entre
% as suas principais observações, eles mostraram que o campo de Ciência da Computação apresentam o padrão de colaboração mais similar para 
% a Matemática do que para a Biologia. Diferente desses esforços, aqui nós focamos no estudo das propriedades do núcleo da comunidade, assim, 
% nossas análises são complementares a essas.
% There are also efforts that attempted to study scientific communities. Particularly, Backstrom \textit{et al.}\cite{Backstrom:2006} studied communities in LiveJournal and
% scientific communities extracted from DBLP to find that the propensity of individuals to join communities and of communities to grow rapidly, depends in subtle ways on the
% underlying network structure. Huang \textit{et al.}~\cite{Huang:2008} used DBLP data to construct a network for the Computer Science field covering research collaborations from
% 1980 to 2005. Among their main observations, they show that the Computer Science field presents a collaboration pattern more similar to Mathematics than to Biology.  
% Different from these efforts, here we focus on studying the properties of the community core, thus, our analyses are complementary to
% theirs.

Quando se trata de identificar núcleos de comunidades, existem várias abordagens que extraem 
o núcleo tendo como base as propriedades topológicas da rede. \cite{Leskovec2010} compararam vários algoritmos 
de detecção de comunidades em vários tipos de rede. Neste trabalho, uma comunidade é definida como nodos que possuem mais ligações
entre si do que com o restante da rede. Os autores apontam que, de forma geral, os algoritmos são otimizados e 
detectam as comunidades efetivamente. No entanto, existem classes de redes em que os algoritmos executam de 
forma subótima. Já \cite{Chakrabarti2006} propuseram um arcabouço baseado em agrupamento hierárquico 
aglomerativo e \textit{k-means} para realizar o agrupamento de dados através do tempo. Eles consideraram dois critérios para avaliação: 
(i) o agrupamento ao longo do tempo não deve apresentar grandes diferenças em relação aos dados globais e 
(ii) o agrupamento não deve mudar drasticamente de uma iteração para outra. Eles construíram uma rede 
a partir de dados do Flickr e constaram que o arcabouço proposto atende aos dois critérios definidos. 
\cite{Hopcroft2004} também propuseram um arcabouço baseado em 
agrupamento hierárquico capaz de identificar comunidades na biblioteca digital CiteSeer. Além dessas comunidades,
eles também apontaram um grupo de artigos que aparece várias vezes nos grupos identificados ao longo do 
tempo, dos quais eles denominaram como núcleo da rede. Tais abordagens não são aplicáveis em nosso 
contexto, já que estamos interessados no estudo das propriedades dos núcleos.

% \cite{Sachan2012}
% ~\citep{Leskovec2010,Chakrabarti2006,Hopcroft2004,Sachan2012}. Particularmente,
% abordagens para identificar o núcleo das comunidades e a caracterização de algumas propriedades dos núcleos obtidos. 



% Finalmente, quando se trata de identificar o núcleo das comunidades, existem muitas abordagens que extraem o núcleo baseado nas propriedades
% da topologia da rede~\citep{Leskovec2010,Chakrabarti2006,Hopcroft2004,Sachan2012}. Particularmente,
% abordagens para identificar o núcleo das comunidades e a caracterização de algumas propriedades dos núcleos obtidos. Tais abordagens
% não e aplicável em nosso contexto, já que nós estamos interessados no estudo de propriedades dos núcleos.
% Finally, when it comes to identifying the community core, there are many approaches that extract the core based on structural properties of the underlying
% network~\cite{Leskovec@www2010,Chakrabarti:2006:EC:1150402.1150467,citeulike:370723,Sachan:2012}.  Particular, 
% Seifi \textit{et al.}~\cite{Seifi:2012:CCE:2187980.2188258} combined four different
% approaches to identify a community core and characterized some properties of the obtained cores. Such approach is not applicable to our context, as we are interested in studying
% network properties of the community core. 