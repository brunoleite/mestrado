\section{Introduction}

Since its beginning, society has been organizing itself into communities, which are groups of individuals with common interests\footnote{http://www.merriam-webster.com/dictionary/community}.
Particularly, the proliferation of new communication technologies based on the Internet has facilitated the rapid formation and growth of online communities~\cite{Kleinberg@cacm2008}. 
Communities exhibit a wide range of characteristics and serve a variety of purposes, from small groups engaged in tightly niche topics such as a very specific scientific community, 
to millions of users linked by an interest such as a community related to a sport team or fans of a celebrity. 

Often, individuals who are socially connected in a tend to share interests and similarities. Although, there are many factors that might determine a community formation
and its growth, there are two main driven forces used to explain similarity in a community formation: influence and homophily. On one hand, influence posits that individuals change
to become more similar to their friends in the community. On the other hand, homophily postulates that individuals create social connections within a community precisely because
they are already similar. Recent efforts have provided quantitative evidences of both forces~\cite{icwsm10cha,crandall.kdd08,Backstrom:2006,influence.correlation.kdd08} and
existing theories~\cite{Rogers.1962,accidental-influential}, models~\cite{kempe03kdd,Kempe05influentialnodes}, and
approaches~\cite{saez-trumper@kdd12,Weng:2010:TFT:1718487.1718520} rely on identifying a group of influential individuals with the power to affect not only the underlying network
structure of a community, but also to interfere on the spread and flow of information within a community. 

In this paper, we take a different perspective and study a complementary problem. Here, we focus on studying the roles that influential individuals from a scientific community play
on evolving properties of such communities. Intuitively, when prolific research leaders decide to join or leave a community, they take with them resources,
experience and students, and they possibly influence other members to do the same, which makes scientific communities very suitable for this kind of study. To construct these
communities we used data from DBLP to identify scientific communities, represented by the main ACM SIG conferences. Then, we propose a strategy to infer the community core, 
the leaders of a given scientific community in a given period of time. Finally, we investigate how aspects of the core impact on the community underlying structure. 

The study of the core of scientific communities is of interest from two different perspectives.  The first is sociological, coming from the necessity to understand how
segments of society evolve as well as to answer longstanding  questions related to the interaction among different types of participant. On the other hand, from a technological perspective,
understanding these aspects is critical not only for link prediction, but also for the design of better recommendation systems. 
Such a study, however, has been difficult as essential components like human connections and a proper definition of leadership is hard to be
reproduced at a large scale within the confines of a research laboratory.

Among our main observations, our results show that members of the community core work as bridges that connect smaller clustered research groups as well as increase the average
degree of the community underlying network, but decrease the overall network assortativeness. More important, we note that variations on the members of the community core tend to be strongly correlated
with variations on these metrics.

The rest of this paper is organized as follows. Section 2 surveys related work. Section 3 describes our strategy and dataset used to construct the 
scientific communities.  Section 4 describes our strategy to compute the community core and Section 5 investigates the
role that these sets of researchers play within their communities.  Finally, Section 6 presents our conclusions and provides directions for future work. 




