Several efforts have been done to understand and model dynamic aspects of 
scientific communities. Despite the great interest, little is known about the role that different members play 
in the formation of the underlying network structure of such communities. In this dissertation, we  
investigate the roles that members of the core of scientific communities play in the formation and 
evolution of the collaboration network structure. To do that, we define a community core based on a metric, 
named \textit{CoScore}, which is an h-index derived metric that captures both, the prolificness and the involvement
of researchers in a community. Our results provide a number of key observations related to community formation and 
evolving patterns. Particularly, we show that members of the community core work as bridges that connect smaller 
clustered research groups. Furthermore, these members are responsible for an increase in 
the average degree of the whole community underlying the network and a decrease on the overall network assortativeness. 
More important, we note that variations on the set of members that form the community core tend to be strongly correlated with 
variations on these metrics. We also show that our observations are important to characterize the role of key 
members on the formation and structure of these communities.

% \keywords{Scientific Communities, Core Community, Community Evolution}

% Diversos esforços têm sido realizados para compreensão e modelagem de
% aspectos dinâmicos de comunidades científicas. Apesar do grande interesse, pouco se sabe
% sobre o papel que diferentes membros têm na formação da estrutura topológica da rede 
% dessas comunidades. Nesta dissertação, investigamos o papel 
% que os membros do núcleo de comunidades científicas têm na formação e evolução de sua rede de 
% colaboração. Para isso, definimos o núcleo de uma comunidade com base em uma 
% métrica, denominada \textit{CoScore}, derivada do índice~h que captura 
% tanto a prolificidade quanto o envolvimento dos pesquisadores na comunidade. 
% Nossos resultados subsidiam uma série de observações importantes relacionadas à 
% formação e aos padrões de evolução das comunidades. Particularmente, mostramos que os membros 
% do núcleo das comunidades atuam como pontes que conectam pequenos grupos de pesquisa. 
% Além disso, esses membros são responsáveis pelo aumento do grau médio de toda a rede
% que representa a comunidade e pela redução de sua assortatividade. Mais importante, 
% notamos que variações no conjunto de membros que compõem o núcleo das comunidades tendem a ser fortemente 
% correlacionadas com variações dessas métricas. Mostramos ainda que nossas observações 
% são importantes para caracterizar o papel dos principais membros na formação e na estrutura das comunidades.