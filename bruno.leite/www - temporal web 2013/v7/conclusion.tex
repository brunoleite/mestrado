
\section{Conclusions}

In this work we provide a deep investigation of the roles that members of the core of scientific communities play in the coauthorship network structure formation and evolution.
Our effort builds upon previous existent studies as it focuses on the core community instead of analyzing the evolutionary aspects of entire communities.  To do that, we defined a
community core based on a new metric, namely \textit{core score}, an h-index derived metric that captures both, the prolificness and the involvement of researchers in a community. Our analysis
suggests that the members of the core community work as bridges that connect smaller clustered research groups. Additionally, we noted that the members of the core community tend to
increase the average degree of the network and decrease the assortativeness. More important, we noted that variations on the members of the community core are strongly correlated
with variations on network properties.  Our study also highlights the importance to study the members of the community core and we hope that our observations might inspire future
community formation models.

As future work, we would like to extend and apply our analysis of the community core to other contexts such as massive multiplayer games and on-line social networks.



