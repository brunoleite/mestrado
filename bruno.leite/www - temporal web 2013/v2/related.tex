\section{Related Work}






Studies has been done to understand the structure of social networks, some studies has focus at temporal evolutions. 
In this terms, \cite{Viswanath:2009} shows a studies where they used two Facebook's dataset. 
The first dataset is about user's profile information, in other words, it is just the user's public information 
and friendship. 
The second dataset contains the data of interaction between the user's Facebook. In Facebook, a user's friends 
can post comment's to the user's wall, these comments appear on the user's wall and can be seen  by others who 
visit the user's profile. 
In these way, \cite{Viswanath:2009} modeled the first dataset as a undirectional graph and the second as a directional 
graph to analysis.
\cite{Viswanath:2009} shows that the second dataset contains information about link's disappearance, because user 
usually stopped to interact, but hardly he removes a friendship tie, it is the cause of the first dataset do not have 
this information. Other cause of the first dataset do not represent the real world is that some users just accept a 
friendship invite as courtesy. In this paper also is showed a measures to show the link's evolutions on the time, 
the resemblance.
\cite{Viswanath:2009} showed that, in the second dataset, though there is high churn in the user pairs that interact 
over time, many of the global structural properties remained relatively constant over time.\\
\\
Is showen in another study \cite{Kumar:2006} the components evolutions of two social networks of a big company. These 
study analysis the components of three ways. The analysis is done about nodes that have degree one, intermidiate components 
(components that are not the gigant component and has more than one degree) and the gigant component. In the intermidiate 
components is showed the concept of star nodes, this nodes are importants to the social networks, because they can make 
the social network grow. Also proposed is mathematic model that making the prediciton of behavior identified in these 
study. In this study \cite{Kumar:2006} is done a characterize about the nodes type too where they can be passive, 
linkers and invites.
\\
\cite{Sun:2012} \\
\cite{Ducheneaut:2007} \\
\cite{Backstrom:2006} \\
\cite{Patil:2012} \\
\cite{Lopes:2011} \\
\cite{Sachan:2012} \\
\cite{Leskovec:2005} \\
\cite{Xu:2010} \\
\cite{Willinger:2010} \\
\cite{Wu:2009} \\
\cite{Huang:2008} \\
\cite{Garg:2009} \\
\cite{Leskovec:2008} \\



