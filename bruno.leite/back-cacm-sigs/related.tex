\section{Related Work}


There has been a number of recent efforts that attempt to analyze community structure and network evolution.  Particularly, Kumar \textit{et al.}~\cite{Kumar:2006} analyzed two large networks to
find a segmentation of these networks into singletons, isolated communities, a giant component. Then, they propose a network growth model able to generate networks with similar
characteristics.  Ducheneaut \textit{et al.}~\cite{Ducheneaut:2007} extracted and characterized explicitly created communities from the World of Warcraft, a massive multiplayer game.
Complementarily, Patil \textit{et al.}~\cite{Patil:2012} analyzed and modeled factors that make users to leave or join on-line gaming communities.  Viswanath \textit{et
al.}~\cite{Viswanath:2009} studied the evolution of activity between users in Facebook and found that that links in the activity network tend to come and go rapidly over time, and
the strength of ties exhibits a general decreasing trend of activity as the social network link ages.

In terms of models for network dynamics, Leskovec \textit{et al.}~\cite{Leskovec:2005} investigated a wide range of real graphs to show that graphs densify over time, with the
number of edges growing super linearly in the number of nodes and that the average distance between nodes often shrinks over time. Based on these observations, they develop a graph
generation model that incorporates such properties.  More recently, Leskovec \textit{et al.}~\cite{Leskovec:2008} presented a detailed study of network evolution by analyzing four
large on-line social networks.  They investigated a wide variety of network formation strategies to show that edge locality plays a critical role in the evolution of networks. Based on
this observation, they developed a model of network evolution, in which nodes arrive at a pre-specified rate.  Differently from the above efforts, our work focuses on community
properties and the roles that community leaders play in the underlying network structure.

There are also efforts that attempted to study scientific communities. Particularly, Backstrom \textit{et al.}\cite{Backstrom:2006} studied communities in LiveJournal and
scientific communities extracted from DBLP to find that the propensity of individuals to join communities and of communities to grow rapidly, depends in subtle ways on the
underlying network structure. Huang \textit{et al.}~\cite{Huang:2008} used DBLP data to construct a network for the Computer Science field covering research collaborations from
1980 to 2005. Among their main observations, they show that the Computer Science field presents a collaboration pattern more similar to Mathematics than to Biology.  
Different from these efforts, here we focus on studying the properties of the community core, thus our analyses are complementary to
theirs.

Finally, when it comes to identifying the community core, there are many approaches that extract the core based on structural properties of the underlying
network~\cite{Leskovec@www2010,Chakrabarti:2006:EC:1150402.1150467,citeulike:370723,Sachan:2012}.  Particular, 
Seifi \textit{et al.}~\cite{Seifi:2012:CCE:2187980.2188258} combined four different
approaches to identify a community core and characterized some properties of the obtained cores. Such approach is not applicable to our context, as we are interested in studying
network properties of the community core. 


