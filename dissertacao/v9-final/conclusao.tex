\chapter{Conclusões e Trabalhos Futuros}\label{conclusao}

\section{Revisão do Trabalho}

Nesta dissertação apresentamos a caracterização de várias comunidades científicas presentes na DBLP, 
uma biblioteca digital da área de Ciência da Computação da qual extraímos e construímos 22 redes 
de colaboração científica. Em seguida, realizamos uma investigação profunda sobre os papéis que os membros 
dessas comunidades desempenham na formação e evolução topológica dessas redes.

Trabalhos anteriores estudam comunidades inteiras e são baseados em abordagens algorítmicas, 
tais como agrupamento hierárquico e \textit{k-means}, que buscam identificar grupos de nodos mais agrupados entre si 
do que ao restante da rede ou nodos que possuem alguma característica topológica. Diferentemente, focamos aqui em identificar os 
principais membros de uma comunidade que chamamos de núcleo da comunidade. Para determinar este núcleo, primeiramente 
definimos uma nova métrica chamada de \textit{CoScore}, derivada do índice~h. Como nossas análises são 
focadas na evolução temporal dessas comunidades, realizamos um estudo para estimar o índice~h de um dado pesquisador ao 
longo do tempo. Utilizamos dados coletados pelo projeto SHINE e comparamos várias estratégias de evolução para este 
índice, demonstrando que a estratégia final escolhida possui forte correlação positiva com os respectivos valores 
do Google Citations, uma ferramenta largamente utilizada pela comunidade 
científica para estimar o índice~h de um pesquisador. Assim, nossa métrica é capaz de capturar 
tanto a prolificidade de um pesquisador, quanto o seu envolvimento em uma comunidade ao longo do tempo. 

Após quantificar a importância de cada membro de uma comunidade, definimos dois importantes limiares 
para nosso estudo, o tamanho do núcleo da comunidade e o tamanho da janela temporal, sendo estes 
10\% e 3 anos, respectivamente. Nossos resultados indicam que membros dos núcleos são 
pesquisadores que contribuem ativamente com publicações para suas comunidades e possuem posição de 
destaque nelas. Desta forma, mostramos que vários desses membros foram premiados por suas contribuições 
à sua comunidade, incluindo até mesmo alguns que receberam o ACM \textit{A.M. Turing Award}.

Nossas análises mostraram, ainda, que o coeficiente de agrupamento dos membros do núcleo tende, de forma geral, a ser 
menor do que os dos demais membros da rede. No entanto, o valor da métrica \textit{betweenness} dos membros 
do núcleo tende a ser consideravelmente maior do que a dos não membros, indicando 
que membros dos núcleos atuam como pontes que interligam pequenos grupos de pesquisa. Além disso, 
observamos que os membros dos núcleos tendem a aumentar o grau médio, diâmetro, caminho mínimo médio e o 
tamanho do maior componente conectado da rede, e diminuir a assortatividade e o coeficiente de agrupamento.
Mais importante, observamos que as variações observadas nos conjuntos de membros dos núcleos das comunidades estão fortemente correlacionadas 
com as variações nas propriedades topológicas da rede. 

Finalmente, fornecemos uma representação visual dessas comunidades que reforça nossas análises e observações. 
Nossos resultados também destacam a importância de se estudar a relevância dos membros dos núcleos das comunidades e esperamos 
que nossas observações possam inspirar futuros modelos de formação de comunidade.


% Neste trabalho apresentamos uma investigação profunda dos papéis que os membros do núcleo das 
% comunidades científicas desempenham na formação e evolução da estrutura da rede de coautoria. Nosso esforço 
% se baseia em estudos anteriores existentes, uma vez que se concentra no núcleo de comunidades, em vez de 
% analisar os aspectos evolutivos de comunidades inteiras. Para isso, definimos um núcleo para as comunidades 
% baseado em uma nova métrica, ou seja, \textit{CoScore}, uma métrica derivada do índice~h que captura 
% tanto, a prolificidade, quanto o envolvimento de pesquisadores em uma comunidade. Nossas análises sugerem que membros 
% do núcleo da comunidade atuam como pontes que conectam pequenos grupos de pesquisa. Além disso, observamos que 
% os membros do núcleo das comunidades tendem a aumentar o grau médio da rede e diminuir a assortatividade. 
% Mais importante, observamos que as variações nos membros do núcleo da comunidade estão fortemente correlacionadas 
% com as variações nas propriedades de rede. Nosso estudo também destaca a importância de estudar os membros 
% do núcleo da comunidade e esperamos que nossas observações possam inspirar futuros modelos de formação da comunidade.

\section{Trabalhos Futuros}

A partir dos resultados do nosso estudo algumas oportunidades de trabalhos futuros foram identificadas e são listadas a seguir:

\begin{itemize}
  \item \textbf{Aplicação do estudo em outros contextos.} Nossas análises do núcleo das comunidades são aplicáveis a outros 
		contextos, como jogos multijogador massivo, OSNs e repositórios de outras 
		naturezas, como filmes e livros.
  \item \textbf{Utilização de outras métricas de prolificidade.} Existem outras métricas capazes de medir a 
		prolificidade de um pesquisador além do índice~h que poderiam também serem utilizadas no cálculo do 
		\textit{CoScore}, como o índice~g \citep{Egghe2006}.
  \item \textbf{Avaliação do \textit{CoScore} em outros contextos.} Nossa métrica quantifica a importância dos 
	        membros das comunidades. Desta forma, também poderíamos utilizá-la em outros contextos, e.g., para predição 
	        de \textit{links} e em sistemas de recomendação.
  \item \textbf{Geração de modelos de formação de comunidades.} O \textit{CoScore} pode ser combinado a outras
	        métricas para mapear o comportamento de como nodos e arestas surgem na rede, possibilitando a 
	        geração de modelos de formação de comunidades, conforme resultados prévios apresentados por \cite{Leskovec2005, Leskovec2008}.
  \item \textbf{Aplicação da abordagem proposta para o estudo de \textit{clusters}.} Vários trabalhos na literatura utilizam
	        abordagens algorítmicas para identificação de \textit{clusters} e de nodos importantes na topologia da rede. No entanto, 
	        essas abordagens possuem um custo computacional considerável. Assim sendo, seria interessante aplicar a nossa abordagem 
	        em estudos semelhantes.
  \item \textbf{Análise do impacto da migração de pesquisadores entre comunidades.} A migração dos membros do núcleo de uma comunidade
		pode ser utilizada para prever o sucesso ou declínio dessa comunidade. Sendo assim, nossa métrica poderia ser aplicada
		em estudos que visem caracterizar a migração de membros de um comunidade ou até inspirar modelos capazes de predizer tais migrações.
\end{itemize}


% Como trabalho futuro, gostaríamos de estender e aplicar nossa análise do núcleo da comunidade para 
% outros contextos, como jogos multijogador massivo, redes sociais online e repositórios de outras naturezas, como filmes e livros.

% In this work we provide a deep investigation of the roles that members of the core of scientific communities play in the coauthorship network structure formation and evolution.
% Our effort builds upon previous existent studies as it focuses on the core community instead of analyzing the evolutionary aspects of entire communities.  To do that, we defined a
% community core based on a new metric, namely \etxtit{core score}, an h-index derived metric that captures both, the prolificness and the involvement of researchers in a community. Our analysis
% suggests that the members of the core community work as bridges that connect smaller clustered research groups. Additionally, we noted that the members of the core community tend to
% increase the average degree of the network and decrease the assortativeness. More important, we noted that variations on the members of the community core are strongly correlated
% with variations on network properties.  Our study also highlights the importance to study the members of the community core and we hope that our observations might inspire future
% community formation models.
% 
% As future work, we would like to extend and apply our analysis of the community core to other contexts such as massive multiplayer games and on-line social networks.